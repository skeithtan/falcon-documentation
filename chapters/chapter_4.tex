\chapter{The Existing System}

\section{Organization Profile}

\subsection{History}
The Philippine Normal School is the first high school established during the American colonial era in September 1, 1901. By June 18, 1949, the school was officially a college. The school's name was then changed to Philippine Normal University (PNU), and was officially allowed to give students undergraduate and graduate degrees.

\subsection{Nature of the Philippine Normal University}
The nature of Philippine Normal University is to gain and disseminate knowledge among people. PNU has a variety of programs that students may enroll depending on their preference. It is then the duty of PNU to educate these students so that they may be equipped with the proper knowledge when entering the industry after graduation.

 According to the official website, Philippine Normal University's curriculum is divided into three (3) colleges and three (3) institutes. The College of Teacher Development, and College of Graduate Studies and Teacher Education Research are highlights of the university. Furthermore, the graduate studies is the largest graduate school for education. The College of Teacher Development handles the undergraduate curriculum. The duty of these 2 colleges is to train students to become effective teachers by enhancing students' educational research skills and leadership skills. The College of Flexible Learning and e-PNU is responsible for managing the online undergraduate and graduate studies of the university. 

The Institution of Knowledge Management is responsible for the constant refreshing of the knowledge base of the university. This could be done in several methods such as management of organizational structure, and management of members. The Institute of Physical Education, Health, Recreation, Dance and Sports is responsible for the degrees involving physical education, and the Institute of Teaching and Learning is the earlier levels of education such as grade school, and high school.

Philippine Normal University also holds different departments to organize the different programs based on their nature such as Faculty of Arts and Languages, Faculty of Behavioral and Social Sciences, Faculty of Education Science, Faculty of Science, Technology, and Mathematics, Institute of Physical Education, Health, Recreation, Dance and Sports, School of Information and Knowledge Management.

Faculty of Arts and Languages is the main priority of this project. They are responsible for handling courses such as Literature Education, English Education, Music and Arts education, and Filipino Education. They currently have about 30 to 40 faculty members, and 1000 students; however, the numbers may vary.

\subsection{Organizational Chart}
Refer to appendix \ref{appendix:OrgChart} for the diagram of the organizational chart.

\section{Description of the System}

\subsection{Gathering and Processing Faculty Information}

FAL collects forms that contain information from faculty members, including but not limited to their name, age, expertise, researches, seminars, and training. The forms gathered from faculty members are produced by different offices; there is no standard faculty form. The information gathered is useful in faculty loading, as it is necessary for FAL's administration to evaluate the faculty members' constraints (location, time availability, workload) and expertise in assigning them to the classes. FAL is also responsible for tracking the number of activities its faculty members participate in. Members of the faculty are required to accomplish a maximum of three (3) activities per term. The clerk of FAL is responsible for monitoring the faculty members' activities and encoding their updated information. When updating the information of faculty members is necessary, the updates are passed from FAL as recommendations or endorsements to the Office of the Dean, the Office of the Vice President, and finally the President's Office. When requested, usually for the participation of a faculty member in an activity, information about the faculty member will be generated and submitted. These information are also reported at the end of the year. Some recipients of the output of faculty information are the Vice President for Academics, FAL's Dean's Office, the Human Resource Management Office, and the Registrar's Office.

\subsection{Faculty Loading}
Faculty loading starts when the FAL receives the list of available subjects from the registrar's office. This schedule of classes is received one month before the start of classes. The associate dean then checks the list of available faculty members and their expertise. Faculty members have classifications. These classifications are used in order to determine the amount of load each faculty member should be assigned. After checking the faculty list, the associate dean assigns available faculty members to classes listed on the Schedule of Classes (Refer to appendix \ref{appendix:ClassesSchedule}). The assignment of the faculty members is mostly based on previous loading assignments and expertise. Cartolina covered in plastic, along with color coded paper, are used by the associate dean to plot faculty members to their respective time slots (Refer to appendix \ref{appendix:Cartolina}). After making the schedule, the associate dean, along with two other faculty members, hold a meeting with all faculty members. They will discuss whether the faculty members approve of the schedule or not. If the faculty members disapprove the schedule, the associate dean asks them for their time preferences and additional constraints, and holds another meeting to ask for approval again. Once all faculty members approve their schedules, the finalized schedule is signed by the associate dean and then submitted to the registrar.

\subsubsection{Permanent Regular Faculty}
The faculty members that fall under this classification are assigned a regular load of three (3) subjects of nine (9) units in a term.

\subsubsection{Permanent Regular Faculty with a Professor Rank}
Permanent regular faculty members that have a professor rank are given two regular loads of six (6) units, along with one non-teaching load that is equivalent to three (3) units.

\subsubsection{Permanent Regular Faculty with an Administrative Position}
Permanent regular faculty members may also hold an administrative position. Such positions include Dean, Associate Dean, Director, and Vice President. Faculty members under this classification are not given regular loads (known as 'full release'). However, these faculty members have the option to accept additional loads.

\subsubsection{Adjunct Faculty}
Faculty members under this classification are not from the Faculty of Arts and Languages. These members may be given additional loads.

\subsubsection{Part-time Faculty}
These faculty members are not from PNU. Only a maximum of three (3) loads may be assigned to them.

\section{Problem Areas}

\subsection{Faculty Loading}
The process of assigning faculty members to scheduled classes holds different issues. The group analyzed that FAL is having difficulty and delays in generating faculty loads. FAL’s associate dean stated that it takes around 2 weeks to generate the faculty loads. Considering that the schedule of classes is provided to FAL one month before the start of classes, taking additional constraints, revisions, and approval of faculty members into account can become tedious and time consuming for FAL, especially its associate dean. In extreme cases, classes start without any professors assigned to it. Students become victims of FAL’s inability to resolve the faculty assignment on time.

\subsubsection{Method Problem Areas}
Feedback from the faculty members is necessary for the approval of a loaded schedule. However, it takes a varying number of days to gather faculty feedback because it is standard for FAL to hold meetings to discuss the loads of all faculty members. Faculty members are asked individually to approve their assignments.

\subsubsection{Manpower Problem Areas}
Additionally, time availability constraints are frequently added and taken into consideration for faculty members and their assignments. These constraints are limitations for professors wherein certain time slots cannot be assigned because of their unavailability. Constraints occur because of unforeseen circumstances and issues coming from the faculty members themselves, and can appear at any point of the process of making the faculty loads. These constraints may be added to when a faculty member disapproves of the assigned schedule. Having to frequently account for these constraints slows down the process of generating faculty loads because of the domino effect that leads to the tedious reassignment of classes to faculty. Instances can arise wherein a faculty member is unavailable for a certain time slot. The associate dean will have to assign another faculty member to the particular class. In effect, two schedules will be modified, potentially affecting even more professors and their schedules.

\subsubsection{Other Problem Areas}
Finally, FAL is unable to view previous faculty loads. The associate dean currently relies on familiarity with the faculty members and remembrance of previous faculty load assignments. This means that the associate dean does not keep a historical record of faculty assignments, which can aid the associate dean in the process of faculty loading. Furthermore, viewing and tracking history can be useful for the future assignment of faculty members. For instance, historical faculty loads can show different information, including the faculty members’ previous time preferences. Additionally, whether a faculty member should take more or lesser load can be inferred. FAL may want to assign less load to a professor that was overloaded in the previous terms. Although this does not directly cause the difficulty in the generation of faculty loads, it is a notable area of improvement for FAL's faculty loading process.

The two other problem areas of Material and Machine are not applicable for faculty loading. Refer to appendix \ref{appendix:FacultyLoadingIshikawa} for the root cause analysis diagram and appendix \ref{appendix:FacultyLoadingBPMN} for the process diagram.

\section{Opportunities}


\subsection{Faculty Profile}
Faculty profiles are collected by FAL and are submitted whenever an office, institution, or an individual requests for them. These are required for tracking the information and activities of the different faculty members. FAL experiences difficulty in doing so, leaving some faculty members unmonitored. According to the associate dean of FAL, some faculty members travel or conduct research without knowledge of the office. Such information needs to be tracked in order for FAL to evaluate the faculty’s workload, achievements, and overall activity within and outside the university. This is especially because faculty members are required to participate a maximum of three (3) activities per term.

The group discovered that FAL does not centralize the data being collected from faculty members. The associate dean of FAL stated that they only collect the initial forms submitted to them; they do not consolidate what is collected. This means that FAL has to deal with multiple forms instead of accessing what is needed from a single source. Furthermore, because FAL does not centralize the data, faculty profiles are redundantly generated instead of being immediately updated and reported.

Additionally, the associate dean stated that the forms collected from the faculty members are not standard because they come from different offices. This means that the contents of the forms include different information about the faculty members. FAL does not harmonize the data that it gathers about its faculty. It is not a prominent issue for FAL, but it is a crucial area of improvement.