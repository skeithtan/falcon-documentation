\chapter{Methodology}

\section{Agile-Scrum Methodology}
The Scrum methodology is one of the most commonly used Agile methods. Each iteration includes planning, developing, testing, and constant feedback from the customer. Members include the Product Owner, Scrum Master, and the Development Team. Each sprint includes daily Scrum meetings to update each member about progress and possible issues. This methodology is also more adaptable to changes because of the emphasis on iterative processes and feedback from customers.

\subsection{Product Backlog}
The product owner will create a product backlog, which is a list of requirements of the finished product sequenced based on priority. The team contacts the client to gather the requirements then uses Notepad and voice recorders. The new requirements are inputted into Notepad or the voice recorder, then placed later into Trello, which is the master backlog.

\subsection{Sprint Planning}
The team will communicate in person after classes, or through group chats in Slack. It is here where the team plans on which requirement to focus on during the sprint. The team will also verify the scope and goals for that sprint. The output is a sprint plan ready to be put into action.

\subsection{Sprint Backlog}
After the sprint planning, the team will generate a sprint backlog containing all the requirements that need to be accomplished in that sprint

\subsection{Sprint}
The sprint is when the team executes the sprint plan. The team will still coordinate through Slack or in-person throughout the sprint. Each member could also evaluate progress through these communication methods. The sprint should be finished on time and within scope, and should output a product incremental.

\subsection{Scrum Meeting}
The scrum master will host a scrum meeting during the sprint. It is in this meeting where members give progress updates and evaluations. The scrum master will also manage any changes to the sprint. The output would be progress updates, and possibly Sprint plan changes.

\subsection{Sprint Review}
A Sprint review is the evaluation of the sprint. The scrum master updates the team on what parts of the sprint backlog was accomplished or not. The team also discusses which tasks succeeded or problematic, then discuss on how to improve on those tasks. The outputs of the sprint are also demonstrated and reviewed. If the output meets standards of the client, the output of the sprint review is a product increment ready for deployment and an updated product backlog. If the output fails to meet standards, there is a sprint retrospective.

\subsection{Sprint Retrospective}
Sprint retrospective is when the team identifies points for improvement in its development process. The phase returns to the sprint planning phase and repeats the process. The output is an evaluation of needs for improvement for the next sprint.

\section{Research Design and Methodology}
The team will be using interviews with the client, consultations and online research to gather information about the project. The team may travel to PNU to have a personal interview with the client, or contact the client through email. These meetings are when the scope and specific requirements are defined. The information and requirements will then be listed in Notepad, or recorded through voice recording. This information is then analyzed through weekly consultations with their thesis advisor, or through online research.

\section{Consensus}
The team has chosen Scrum because one of the attributes is better customer satisfaction due to the active role of the product owner in the methodology. The team has stable communications with the client. Therefore, the team seeks to take advantage of this by following Scrum to improve customer satisfaction. Scrum is also more adaptable and flexible to change than more traditional methods. This is important because the nature of the project involves constant revisions and modifications throughout development, so the group and the methodology must be able to cope with the changes. Scrum method is also more efficient and faster than traditional methods. A sprint lasts for approximately 2-4 weeks, this will save the development team much needed time. The team also chose Scrum because the team is already familiar with the methodology, as it was used in previous projects. Another attribute is because Scrum saves time (Feliciano-Misla, n.d.). The team has a limited amount of months to accomplish this project. Therefore, the group finds it is necessary to save as much time as possible.

\section{Gantt Chart}
Refer to appendix \ref{appendix:GanttChart} for the Analysis Gantt Chart.
