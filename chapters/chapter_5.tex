\chapter{The Proposed System}

\section{System Description}

\subsection{Inputs and Outputs}

\subsubsection{Faculty Profile}
The faculty profile inputs are the basic information (name, email, date of birth, faculty ID number, sex, employment type), degrees, recognitions, presentations, instructional materials, and subjects of expertise of the faculties. The output of the faculty profile module is the collective faculty profile which contains the details of the faculty members working under FAL. These faculty profiles may be printed, and their content may vary depending on the requesting office.

\subsubsection{Faculty Loading}
Inputs for faculty loading include the availability constraints of the faculty, subjects, and courses. Expertise or specializations of the faculty member, from the faculty profile module, is also included as an input. The system will then automate an output which is the schedule of classes with the assigned professors.

\subsection{Processes}

    \subsubsection{Faculty Loading}
        \paragraph{Adding of Classes and Faculty}
        The first step in the new faculty loading process is the adding of classes and faculty. Classes and their schedules come in the form of a list, and are given to the associate dean by the registrar. These classes and schedules are permanent. The process is performed by either the dean, associate dean, or the clerk. Details of the class being added must be inputted, such as the subject being taught, the class code, the days of the class (Monday-Thursday or Tuesday-Friday), and the time slot. On the other hand, adding faculty members involves selecting the faculty members who will be given assignments for the term. When the faculty members and classes have been added, the faculty members will be able to submit their time preferences.
        
        \paragraph{Scheduling of Faculty}
        After the classes and the faculty members have been added and time preferences have been gathered, exclusively the associate dean can begin assigning faculty members. The scheduling process ends when all classes have been assigned, and ideally when there are no warnings for the assignments of faculty members. When the process is ended, the generated schedules can be viewed by the faculty members.
        
        \paragraph{Gathering of Faculty Feedback}
        When faculty members have viewed their schedules, they can either approve or disapprove them. Faculty members can no longer submit changes to their schedules when they have approved. Should a faculty member disapprove of their schedule, the system prompts them to input their reason for rejection and their new time availability. If the associate dean needs to adjust the schedule accordingly, they may do so. After the schedule is adjusted, all faculty members must approve or disapprove, whether their schedule has been affected or not. Otherwise, the associate dean may proceed with the generated schedule, regardless of faculty disapproval.
        
        \paragraph{Publishing of the Term Schedule}
        After gathering feedback from faculty members, the associate dean can publish the schedule, finalizing it. The generated term schedule will not be submitted to the registrar. It will be kept by FAL for internal documentation, and can be submitted to the dean or the university's vice president for academics.

    \subsubsection{Faculty Profile}
        \paragraph{Creating the Account}
        Adding a faculty member and creating their account on the system is performed by either the dean, associate dean, or clerk. Inputs include the new faculty member's first and last names, email address, and faculty photo to be uploaded. The account then gets created. Afterwards, the faculty's basic details are set, such as their date of birth, sex, faculty identification number (formatted as T-XXX, XXX as a three-digit number), and their employment type (regular, professor, administrative, adjunct, part-time). The last step in creating the faculty account is the provision of a temporary password, which is automatically generated by the system. The dean, associate dean and clerk are held responsible for storing the temporary password in a separate and safe place for handing over to the faculty member. This automatically generated password will be used by the faculty member to log in for the first time.
    
        \paragraph{Setting the Account}
        Upon logging in for the first time through their email and temporary password generated by the system, faculty members are required to change their password. After logging in, the faculty member now gains access to their own profile. This means that the faculty member can also add necessary information about them. Their subjects of expertise are selected by the dean, associate dean, or clerk, while all the other details, such as the degrees, recognitions, presentations, instructional materials, and extension works must be requested for adding.
    
        \paragraph{Requesting for Profile Changes}
        When adding faculty members' degrees, recognitions, presentations, instructional materials and extension works, the dean, associate dean and clerk do not have to go through the process of requesting for changes; the changes in details are immediately in effect. However, faculty members adding such details to their profiles must undergo the process. This is done by filling out the respective forms for the different profile sections and submitting the change request. Faculty members can withdraw or cancel their change request. The dean, associate dean, or clerk then checks for change requests sent by faculty members, and either accepts or rejects them. When a change request is rejected, the user must indicate the reason for rejecting the change request. On the other hand, when the change request is accepted, the changes are immediately in effect.

\subsection{Data Storage Location}
Data to be inputted and generated to and by the system is stored in MongoDB. To ensure documents are accessible only to those permitted, the server employs authorization and authentication techniques in API data access.

\section{System Objectives}

    \subsection{General Objectives}
    The research objective is to alleviate PNU's issues identified in the problem areas of faculty loading and faculty profile through the implementation of an information system.
    
    \subsection{Specific Objectives}
    Specifically, the project aims to:
    
    \begin{itemize}
    \item{To facilitate tracking and management of faculty subject assignments}
    \item{To facilitate creating, updating, and storing faculty profiles}
    \item{Assign subjects to faculty}
    \end{itemize}

\section{System Scope}

    \subsection{Faculty Loading}
    %-- to do
    
    \subsection{My Schedule}
    %-- to do
        
    \subsection{Faculty Profiles}
    The following functions are performed by either the dean, associate dean, or clerk. The said users have a different interface for these functions than the faculty members.
    
    	\subsubsection{Add Faculty}
    	The user can add a faculty profile when a new faculty member joins FAL. Inputs include the name, email, photo, date of birth, sex, identification number, and employment type. The system will automatically generate a temporary password for the faculty member for the first time they will log in. This password must be stored by the clerk separately.
    	
    	%-- Overview
    	\subsubsection{View Faculty Profile}
    	Clicking on a faculty member on the list will display their profile. Other tabs are provided for the user to view other details, such as the faculty member's presentations, instructional materials, extension works, and change requests. The user will also be able to see the basic details of the faculty member, along with their subjects of expertise, degrees, and recognitions.
    	
    	\subsubsection{Update Profile Details}
    	The user can update the selected faculty's details, such as their name, email, faculty ID number, date of birth, photo, sex, and employment type.
    	
    	\subsubsection{Reset Password}
    	When a faculty member's temporary password is not stored elsewhere by the user, or the faculty member physically requests for a new password, the user can reset their password. This will generate a new temporary password for the faculty member to use in logging in to the system. Similar to the initial automatically generated password, this must be stored elsewhere by the user.
    	
    	\subsubsection{Print Faculty Profile}
    	The user can print the faculty profile. By default, the selected faculty member's basic details are included in the document to be printed. However, the user can select further content to include. This is because the content is dependent on the requesting office that the document will be submitted to.
    	
    	\subsubsection{Assign Subject}
    	The user can assign subjects that the selected faculty member is an expert or is experienced in.
    	
    	\subsubsection{Unassign Subject}
    	The user can remove a subject of expertise from the selected faculty member.
    	
    	\subsubsection{Add Degree}
    	The user can add a degree to the selected faculty member. The user is then prompted to input the degree title, year of completion, and level of degree (associate, bachelor, master, or doctorate).
    	
    	\subsubsection{Add Recognition}
    	The user can add a recognition to the selected faculty member. The user is prompted to input the title of the recognition, sponsor, date of recognition, and basis of the recognition (research, scholarship, extension work, or instruction).
    	
    	%-- Presentations
    	\subsubsection{View Presentations}
    	The user can view the different presentations of the selected faculty member.
    	
    	\subsubsection{Add Presentation}
    	The user can add a presentation to the profile of the selected faculty member. Details such as the title of the presentation, sponsor, venue, conference name, presentation date, and duration (in days), are required to be inputted before adding a presentation.
    	
    	%-- Instructional Materials
    	\subsubsection{View Instructional Materials}
    	The user can view the different instruction materials produced by the selected faculty member.
    	
    	\subsubsection{Add Instructional Material}
    	The user can add an instructional material to the profile of the selected faculty member. Details such as the title of the instructional material, year of usage, audience (student or teacher), and medium (print, module, video, slide, digital file, or audio), are required to be inputted before adding an instructional material.
    	
    	%-- Extension Works
    	\subsubsection{View Extension Works}
    	The user can view the different extension works of the selected faculty member.
    	
    	\subsubsection{Add Extension Work}
    	The user can add an extension work to the profile of the selected faculty member. Details such as the title of the extension work, venue, and roles (lecturer, trainer, resource speaker, facilitator, coach, material writer), are required to be inputted before adding an extension work.
    	
    	%-- Change Requests
    	\subsubsection{View Change Requests}
    	The user can view the change requests sent by the selected faculty member. The notification does not disappear until all the change requests are addressed. Furthermore, each change request displays what is being added by the selected faculty member to their profile and when it was sent. The user can either accept or reject change requests.
    	
    	\subsubsection{Accept Change Request}
    	When the user accepts a change request, the requested information is automatically added to the profile of the selected faculty member.
    	
    	\subsubsection{Reject Change Request}
    	However, when the user rejects a change request, the reason for rejection must be indicated before dismissing the change request.
        
    \subsection{My Profile}
    The following functions are performed by faculty members. They have a different interface than the dean, associate dean and clerk and can only view their own profiles.
        
        %-- Overview
        \subsubsection{View Profile}
        By default, when a faculty member logs in, their profile will be displayed. This includes their basic information, employment type, faculty ID number, subjects of expertise, degrees, and recognitions. There are tabs for accessing their presentations, instructional materials, extension works, and submitted change requests.
        
        \subsubsection{Print Profile}
        The user can print their faculty profile. By default, their basic details are included in the document to be printed. Additional information can be included, depending on what the requesting office requires from them.
        
        \subsubsection{Request Degree}
        Adding a degree to the profile must be approved by either the dean, associate dean, or clerk. Therefore, a change request must be submitted, containing the title of the degree, year of completion, and the level of the degree (associate, bachelor, master, or doctorate).
        
        \subsubsection{Request Recognition}
        Adding a recognition to the profile must be approved by either the dean, associate dean, or clerk. Therefore, a change request must be submitted, containing the title of the recognition, sponsor, date, and basis (research, scholarship, extension work, or instruction).
        
        %-- Presentations
        \subsubsection{View Presentations}
        The user can view their different presentations.
        
        \subsubsection{Request Presentation}
        Adding a presentation to the profile must be approved by either the dean, associate dean, or clerk. Therefore, a change request must be submitted, containing the .
        
        %-- Instructional Materials
        \subsubsection{View Instructional Materials}
        The user can view their different instructional materials produced.
        
        \subsubsection{Request Instructional Material}
        Adding an instructional material to the profile must be approved by either the dean, associate dean, or clerk. Therefore, a change request must be submitted, containing the .
        
        %-- Extension Works
        \subsubsection{View Extension Works}
        The user can view their different extension works.
        
        \subsubsection{Request Extension Work}
        Adding an extension work to the profile must be approved by either the dean, associate dean, or clerk. Therefore, a change request must be submitted, containing the .
        
        %-- Change Requests
        \subsubsection{View Change Requests}
        The user can view their change requests submitted for approval by either the dean, associate dean, or clerk. These include details about the change being made, depending on what was requested to be added. The status of approved or rejected change requests can also be seen by the user, including the reason for rejection.
        
        \subsubsection{Withdraw Change Request}
        When a change request with wrong information is submitted, or the user no longer needs the change, the user can withdraw their change request.
        
        \subsubsection{Dismiss Change Request}
        When a change request has either already been approved or rejected by either the dean, associate dean, or clerk, the change request can be dismissed, removing it from the user's display.
    
    \subsection{Subjects}
    The following functions are accessible to the dean, associate dean, and clerk. Faculty members are not allowed to manipulate the subjects.
    
        \subsubsection{View Subjects}
        The user can view a list of subjects, and select one to see its overview. Details such as the full title of the subject, category, description, and expert faculties will be displayed.
        
        \subsubsection{Add Subject}
        The user can add a subject to the system. Details such as the title, subject code, description, category (pedagogical, general education, specialization, elective, or professional education), and expert faculty members must be inputted by the user.
        
        \subsubsection{Update Subject}
        The user can update or change the details of the subject, including the full title, subject code, description, and category (pedagogical, general education, specialization, elective, or professional education).
        
        \subsubsection{Add Expert Faculty}
        The user can add expert faculties to a subject. These faculty members are either experts or experienced in the subject, and therefore are fit to teach the assigned subject.
        
        \subsubsection{Remove Expert Faculty}
        An expert faculty can be removed from a subject, unassigning them.
    
    \subsection{Users}
    %-- to do
    
\section{System Feasibility}

\subsection{Schedule Feasibility}
The team was given January 2018 until April 2018 to create the proposal for the system. The team was then given May 2018 until July 2018 to accomplish the documentation and final software with core modules. Due to the large project scope during the proposal phase, some modules or features were removed or delayed to the the final phase of the thesis in order to make the project feasible in the current schedule.

\subsection{Operational Feasibility}
The system is operationally feasibly due to the cooperation of stakeholders and sufficient manpower. The clients are more than willing to cooperate with the development of the system. The team visited the client multiple times for gathering of information through interviews, where series of questions are asked. The client also responds consistently with emails.

The project team consists of four members, the maximum, permitted number for this project. One member is assigned as the project manager, another is assigned for head of development, and the remaining two members are assigned to documentation. However, there are times when it is necessary for members to temporarily swap roles or contribute to either the system or to the document. The project manager is responsible for overseeing the whole project, as well as contacting the client and faculty advisor, scheduling consultations, and submitting of forms. The head of development is responsible for supervising the overall system. This includes development of the front-end, back-end, database, system architecture, and so on. The last two members are responsible for documenting the system. This includes the main documentation of the system, user manual, and technical manual. These members must ensure the documentation and graphs mirror what is presented in the system.

\subsection{Technical Feasibility}
The minimum required specifications to use the system would be:
\begin{itemize}
    \item{Firefox v. 61}
    \item{Google Chrome v.65}
    \item{Safari v.11}
    \item{Windows 10/Mac OS High Sierra/Linux}
    \item{Intel Core i5 2015}
    \item{8GB RAM}
    \item{20GB free storage}
    \item{Integrated graphics}
\end{itemize}

\subsection{Economic Feasibility}
The system will be economically feasible for both the client and the team. This is due to the fact the software needed to develop and run the programs are free. The project team and client are also equipped with hardware that meet the minimum requirements to run the system. As a result, the client does not need to invest in development costs. The project team also consists of students. Furthermore, it is not necessary for the client to spend more on labor.