\chapter{Review of Related Literature}

\section{Review of Related Concepts}

\subsection{Master Data Management}
Master data management, according to Rouse (2010) \nocite{ref:MasterData}, is a process wherein an organization links all of its necessary file into one file which is called "master file". This master file is then used to show common data among the files or common point reference. Master data management can be used in merging different kinds of forms by showing the common fields and unique fields that are evident. According to Lewis (2009), phases of master data management are discovery (finding data sources), analysis (identifying discrepancies), design (designing metadata), implementation (implementation of metadata) and establish data governance. Moreover, it can be applied in faculty profile. Based from the multiple forms for faculty profile of FAL, master data management can be applied:
\begin{itemize}
\item To find faculty profile related forms from different colleges (discovery)
\item To check what were the common fields evident and what were the unique ones which can be used in making a standardized form (analysis and design)
\item For FAL to have the standardized form for faculty profile (implementation)
\item To manage the usability and security of these data coming from the faculty profile (data governance)
\end{itemize}

\subsection{Resource Management}
In an organization, resource management involves allocating the company's resources as efficient as possible. These resources can include finance, goods, equipment, labor, human skills, time, among others \cite{ref:ResourceManagementBusinessDictionary}. Moreover, it involves planning the right amount of resources to certain tasks so that there will be fewer to no resources that will be wasted and it can avoid, if not lessen problems involving insufficient resources \cite{ref:ResourceManagementShopify}. Life cycle of resource management includes estimation phase, planning phase, execute phase, analysis phase and optimize phase (\textit{Resource Management Lifecycle}, 2017). Resource management plan can be applied to faculty loading of FAL because in faculty loading, it involves estimation of the number of units of a certain professor (estimation), and assigning professors in a specific time slot (planning). Execute phase comes in when a schedule has been made and will be used for the whole term. The schedule generated will then be used and analyzed to know whether or not it is effective and can be applied on the succeeding terms . When it comes to changes of schedule because of unexpected circumstances, FAL should be able to respond in real time so to avoid classes without professors (optimization). With that, some aspects and guidelines of resource management can be used for faculty loading.

\subsection{Personnel Management}
Personnel management, according to Grimsley (n.d.) \nocite{ref:PersonnelManagementGrimsley} and Juneja (n.d.) \nocite{ref:PersonnelManagementJuneja}, involves the management of administration concerning with the work of the employees and their relationship within the organization. Moreover, the objective of personnel management is to assign personnel / employees to certain activities and to ensure that the employees have a good relationship with each other. Faculty loading can be related to personnel management since its process involves the associate dean assigning faculty members to certain subjects. In addition with faculty loading, the associate dean also ensures that the faculty members also have a good relationship with each other with regards to the schedule of teaching. This is why some aspects of personnel management can be applied to faculty loading.

\subsection{Faculty Loading}
Faculty loading is the distribution and scheduling of subject loads to each faculty member based on their expertise and constraints. These constraints may include travel time, studies/work outside the university, amount of load, and many others. The faculty loading schedule is planned by the associate dean, and requires the approval of respective faculty members. If the proposed schedule could not accommodate that faculty member, the schedule will be revised until every faculty member approves it. The objective of faculty loading based from FAL is to have a finalized schedule with each class having a professor before the term starts.

\subsection{Faculty Profile}
Faculty profile involves the gathering and storage of faculty members' information. This includes name, age, expertise, researches, and so on. PNU needs the faculty profile for decision making, such as faculty loading. FAL has a policy that each faculty member is required to do three (3) activities per term. These tasks include participating in researches, attending seminars abroad, or hosting a training seminar. Whenever a professor will conduct / participate researches, seminars, or training, that professor's faculty profile will be used for others to recognized him / her. This is why the information coming from faculty profile needs to be reported when necessary.

\subsubsection{Activity}
According to the Cambridge English Dictionary (n.d.) \nocite{ref:Activity}, an activity means the doing to something. In PNU's context, activities are the tasks that faculty members are required to do per term. FAL requires faculty members to accomplish three (3) activities per term. These may include research papers, seminars, and hosting trainings.

\subsubsection{Data Centralization}
Data Centralization is when data is gathered, stored, and updated in one place that can be accessed through several points \cite{ref:DataCentralization}. The benefit of data centralization include quicker searches, less complexity in the IT infrastructure, and reduced redundancy. As for FAL's case, since its faculty profile is based from different faculties and organization, this led them to have difficulties in coming up with reports for its professors. Having the faculty profile centralized in FAL, FAL can easily access its professors' profiles so that necessary reports can be easier to generate.

\section{Review of Related Systems}

\subsection{Elements Sympletic}
Elements, developed by Symplectic, is a software system that is used by research institutions to analyze, collect, and conduct research and scholarly activities. It is widely used by administrators, researchers and other employees of organization in order to gather and analyze information about their work and make these information reusable for future purposes. It has features such as showcasing of academic achievements, reporting and analyzing both institutional and faculty activities \cite{ref:Elements}. Refer to appendix \ref{appendix:ElementsSympletic} for sample screenshots.

\subsection{SudoSchool}
SudoSchool is a cloud-based system that caters to school management including teachers, parents / students and alumni with its various web-based functionalities. It has features such as managing employees and their profiles, customization of reports and external portal for alumni, managing the admission and information of students, fee and library management. Moreover, SudoSchool has a mobile platform which can help users have ease in accessing desired information and increase productivity \cite{ref:SudoSchool}. Refer to appendix \ref{appendix:SudoSchool} for sample screenshots.

\subsection{Yale University: Faculty Information System}
The faculty information system of Yale University is a web-based online system that support annual administrative events such as leave requests, promotion reviews and annual raise process. In addition, it also provides a secure tool that allows broad use across Yale and it serves as a single reference point for faculty appointments, teaching, degrees, leaves and committee assignments. 

\subsection{Doodle}
Doodle is an online tool that is used for scheduling date and time of planned activities. Its features include calendar integration for ease in creating and participating in polls, personalized dashboards that allow users to manage polls, and a mobile application for easier accessibility. Doodle focuses on polling which can let users create poll (contains a list of schedules) then allows other participants to give feedback on what option works best. Refer to appendix \ref{appendix:Doodle} for sample screenshots.

\subsection{De La Salle University Faculty Loading System}
The faculty loading system is specialized for DLSU which has features such as creating course flowchart, dashboard for notifications, creating classes and its schedules, pre-autoloading activities, scheduling special load. Professors can also edit and view their loads, and view collision resolution, deadlock resolution, flowcharts and room assignments. 

\begin{table}
\centering
\caption{Comparison of related systems}
\begin{tabular}{p{1.5in} | p{2in} | p{2in}}
\hline
                                                              & \textbf{Strengths}                                                                                                                                                                                                                                                                                                                     & \textbf{Weaknesses}                                                                                                                                                                                                                          \\ \hline
\textbf{Element Symplectic} & • Provides a database using a simple or advanced criteria for publication metadata, colleague, research, teaching or professional activities and other areas of interest.

• Collect and reuse information pertaining to scholarly output to create a rich online profile for viewing by peers and other departments within the institution.~ & 
• Focuses in the research field of an institution but not in managing the operations of the institution. 

• Because it is a 3rd party software, there are some functions that are unusable and cannot cater the needs of the institution\\ \hline
\textbf{SudoSchool}                                                    & 
• It is has a lot of modules that caters most needs in managing faculty, students, parents, alumni and operations.                                                                                                                                                                                                                        & 
• Most of its features is all about managing faculty, students, alumni and the like but it does not focus on the research side of the institution for improvements of the curriculum.

• The user interface appears outdated.                \\ \hline
\textbf{Yale University: Faculty Information Systems}                  & 
• Has a faculty module that supports administrative events.It can access faculty information pulled from several official university database.                                                                                                                                          & • Very little is known about this system outside of the Yale University because of security concern. \\
\hline
\textbf{Doodle}                  & 
• Simple to use features that focuses on scheduling. In addition, there is no subscription fees and it is also easy to access because of its mobile feature.                                                                                                                                          & • Only focuses on scheduling and most of its features requires manual intervention. \\
\hline

\textbf{DLSU Faculty Loading System}                  & 
• Has a lot of features that focuses on scheduling. In addition, it is specialized for DLSU which caters most of the needs of DLSU with regards to faculty loading.                                                                                                                              & • Since it is specialized only for DLSU, most probably it can't be used for other universities.\\
\hline

\end{tabular}
\end{table}

\clearpage

\section{Review of Related Methodologies}

\subsection{Scrum}
Scrum is an Agile methodology that focuses on an iterative pattern which includes planning, developing, testing and constant feedback from the customer. Each iteration has daily scrum meetings in order to update each member about the progress of the project. In scrum methodology, members include Product Owner, Scrum Master and the Development Team. Product Owner is the one who is responsible for relaying the requirements and purpose of the product to the Scrum Master and the development team. Scrum Master on the other hand is the facilitator of the development team who ensures that the team is following the steps of scrum methodology while the development team are the individuals who has the skill to submit the deliverables and in the presented time-frame of the scrum \cite{ref:ScrumMethodology}. See appendix \ref{appendix:Scrum}. The team decided to use Scrum because the team is already familiar with this methodology the most. It also involves constant revisions and modifications, which Scrum takes into account. Most importantly, one of the unique attributes of Scrum is better client satisfaction because the product owner plays an active role in the methodology. Since the group has stable communication with the client, the team seeks to take advantage of the Scrum attribute. Another attribute is because Scrum saves time \cite{ref:ReasonsScrum}. The team has a limited amount of months to accomplish this project. Therefore, the group aims to save as much time as possible.

\subsection{Extreme Programming}
According to McLaughlin (n.d.) \nocite{ref:ExtremeProgramming}, Extreme Programming is an Agile methodology that focuses on constant planning, testing, and feedback. Clients play an active role in this methodology by providing constant feedback, as small software is being given in small intervals. The functionalities are defined based on the user stories given by the client. Then release planning is done for the prioritized user story. The iteration is when the actual development of the functionality defined by the user story occurs. After the functionality is done, it is tested by the client. The iteration repeats itself until the functionality is accepted by the client, to which it is now released. After the functionality is completed, the process restarts until every user story is complete. See appendix \ref{appendix:ExtremeProgramming}. However, the group decided not to pursue Extreme Programming because none of the members have experience with this methodology. Given the significance of this project, the group has decided to continue with Scrum. Extreme Programming also focuses on simplifying design to save time. However, this could lead to issues later in the project due to not following proper design techniques and outputs.\cite{ref:ExtremeProgrammingDisadvantage}. Given the scale of this project, design is an important factor. The group needs a methodology that emphasizes design more.

\subsection{Kanban}
Kanban is an Agile methodology that emphasizes constant delivery of output while not overworking the team. The 3 main principles of Kanban are workflow, work in progress, and enhance flow. Workflow means to plan what work to accomplish. Work in progress means to balance the amount of work, so that teams may make good progress while not being overworked. Enhance flow means to select the next highest priority in the backlog after the previous task is accomplished \cite{ref:ExtremeProgramming}. Kanban is beneficial because it is adaptable to changes in priorities, faster feature delivery, and constant feedback. See appendix \ref{appendix:Kanban}. The group considered choosing Kanban or Scrum. However, due to Scrum being the most popular Agile methodology \cite{ref:ScrumPopular}, the group decided to follow Scrum.

\subsection{Crystal}
Crystal is one of the many examples of Agile methodologies. Unlike other methodologies, Crystal focuses more on the people involved rather than tools \cite{ref:Crystal}. The team is ranked by color depending on how many members there are. Generally, more members means a more complex approach to a project. For example, clear means 8 people, yellow means 10-20 people, orange means 20-50 people, and so on. See appendix \ref{appendix:CrystalTeamColors}. Crystal has seven core properties, frequent delivery means the constant delivery of modules in the project, reflective improvement means that there is always room for improvement, osmotic communication is the spreading of information, even when the person is not directly communicated, personal safety means that each member of the team is in a comfortable and safe working environment, focus is that each member knows the task that they are responsible for, easy access to expert users means that the team has constant communication with the actual users, and technical environment means the software tools that fit their project. See appendix \ref{appendix:CrystalProperties}. The team decided not to use Crystal because there are only four members. Since Crystal mainly focuses on members, it is unnecessary to use this methodology for a small group.

\subsection{Feature-Driven Development}
FDD is an Agile methodology that focuses on the features of the project \cite{ref:FDD}. FDD starts with developing an overall model, which is when the team creates a general requirements to act as a basis on what needs to be accomplished. The next phase is to create a list of features, this is where they create a more specific list of requirements for the system. Then the next phases are where the bulk of the work for FDD is performed. The team must then plan, design, and build by feature. It is important in FDD that the team work by feature to prevent conflicts and inefficiencies. As opposed to moving back and forth between features, as seen in traditional methodologies. See appendix \ref{appendix:FDD}. However, the team decided not to use FDD because it is better suited for larger development teams. It would be difficult to use FDD in projects with small teams and tight deadlines. \cite{ref:FDDDisadvantage}

\subsection{Lean}
The core of lean is to create maximum customer value while minimizing resource expenses \cite{ref:Lean}.  This means to deliver quality products or services on time without wasting any resources. Lean starts with first identifying the value based on the perspective of the customer. This leads mapping the steam value, which specifies the steps in the value stream and eliminate unnecessary steps. Afterwards is the create flow phase, this is where value-creating steps are added so that the product may reach the customer more efficiently. The next phase is when the customer pulls the value of the new flow. Then lastly, the seek perfection phase is when the team identifies any points for improvement that could reduce more costs. The process would then start again. The team decided not to use Lean as minimizing resource expenses is not a main concern for the group. See appendix \ref{appendix:Lean}. 

\section{Synthesis}
The methodologies reviewed in this chapter are the most similar to the consensus methodology, Scrum. Scrum and extreme programming show resemblance; both of them focus on an iterative pattern of planning, developing and testing that emphasize constant customer feedback. Kanban and Crystal  focuses on team dynamics and values the overall workflow of the team. Feature Driven Development focuses on the features of the project while Lean emphasizes maximizing customer value and minimizing resource expense. Since Scrum is more adaptable and flexible, it fits to the project because the project involves constant revisions and modifications throughout the development process. 